\hypertarget{newAPI_newAPI_intro}{}\section{Introduction}\label{newAPI_newAPI_intro}
With version 2.\+2 we introduce the new A\+PI that will take over the old one in the future version 3.\+0. By then, backwards compatibility will be broken, and third party applications using R\+N\+Alib need to be ported. This switch of A\+PI became necessary, since many new features found their way into the R\+N\+Alib where a balance between threadsafety and easy-\/to-\/use library functions is hard or even impossible to establish. Furthermore, many old functions of the library are present as slightly modified copies of themself to provide a crude way to overload functions.

Therefore, we introduce the new v3.\+0 A\+PI very early in our development stage such that developers have enough time to migrate to the new functions and interfaces. We also started to provide encapsulation of the R\+N\+Alib functions, data structures, typedefs, and macros by prefixing them with {\itshape vrna\+\_\+} and {\itshape V\+R\+N\+A\+\_\+} , respectively. Header files should also be included using the {\itshape Vienna\+R\+N\+A/} namespace, e.\+g. 
\begin{DoxyCode}
\textcolor{preprocessor}{#include <\hyperlink{fold_8h}{ViennaRNA/fold.h}>}
\end{DoxyCode}
 instead of just using 
\begin{DoxyCode}
\textcolor{preprocessor}{#include <\hyperlink{fold_8h}{fold.h}>}
\end{DoxyCode}
 as required for R\+N\+Alib 1.\+x and 2.\+x.

This eases the work for programmers of third party applications that would otherwise need to put much effort into renaming functions and data types in their own implementations if their names appear in our library. Since we still provide backward compatibility up to the last version of R\+N\+Alib 2.\+x, this advantage may be fully exploited only starting from v3.\+0 which will be released in the future. However, our plan is to provide the possibility for an early switch-\/off mechanism of the backward compatibility in one of our next releases of Vienna\+R\+NA Package 2.\+x.\hypertarget{newAPI_newAPI_changes}{}\section{What are the major changes?}\label{newAPI_newAPI_changes}
...\hypertarget{newAPI_newAPI_porting}{}\section{How to port your program to the new A\+PI}\label{newAPI_newAPI_porting}
...\hypertarget{newAPI_newAPI_examples}{}\section{Some Examples using R\+N\+Alib A\+P\+I v3.\+0}\label{newAPI_newAPI_examples}
Examples on how to use the new v3.\+0 A\+PI can be found in the \hyperlink{mp_example_examples_c_new_API}{Using the \textquotesingle{}new\textquotesingle{} v3.\+0 A\+PI} section. 